\documentclass[12pt,ngerman]{beamer}

\usepackage[utf8]{inputenc}
\usepackage[T1]{fontenc}
\usepackage{booktabs}
\usepackage{babel}
\usepackage{graphicx}
\usepackage{csquotes}
\usepackage{xcolor}
\usepackage{listings}
\usepackage{mdframed}

\definecolor{hellgelb}{rgb}{1,1,0.8}
\definecolor{lightgelb}{rgb}{1,1,0.8}
\definecolor{colKeys}{rgb}{0,0,1}
\definecolor{colIdentifier}{rgb}{0,0,0}
\definecolor{colComments}{rgb}{1,0,0}
\definecolor{colString}{rgb}{0,0.5,0}


\usepackage[]{textcomp}
\lstset{%
    float=hbp,%
    basicstyle=\ttfamily\footnotesize, %
    identifierstyle=\color{colIdentifier}, %
    keywordstyle=\color{colKeys}, %
    stringstyle=\color{colString}, %
    commentstyle=\color{colComments}, %
    alsoletter={\_},
	language= {Python},%
    columns=flexible, %
    tabsize=2, %
    morekeywords={read_csv,to_datetime,merge,to_excel,join,concat,pairplot,vq,kmeans,savefig,Series,axis,DataFrame,index,to_frame,loc,iloc,idx,mean,describe,std,count,},%
    frame=single, %
    extendedchars=true, %
    showspaces=false, %
    showstringspaces=false, %
    numbers=left, %
    numberstyle=\tiny, %
    upquote=true,
    breaklines=true, %
    backgroundcolor=\color{yellow!15}, %
    breakautoindent=true, %
    captionpos=b%
}


\lstset{literate=%
    {Ö}{{\"O}}1
    {Ä}{{\"A}}1
    {Ü}{{\"U}}1
    {ß}{{\ss}}1
    {ü}{{\"u}}1
    {ä}{{\"a}}1
    {ö}{{\"o}}1
    {~}{{\textasciitilde}}1
}

\newcommand{\pyc}[1]{\lstinline[language={Python}]{#1}}


\author{Dr. Uwe Ziegenhagen}
\title{Python \& \LaTeX}
\subtitle{Dante e.V. Frühjahrstagung 2017}
\date{22. März 2017}

\begin{document}

\begin{frame}

\maketitle

\end{frame}


\begin{frame}
\frametitle{Überblick}
\framesubtitle{Was machen wir heute?}

\begin{itemize}
\item Python Grundlagen
\item Python in \LaTeX\ Dokumenten
\item Erzeugung von \LaTeX\ Dokumenten
\end{itemize}
\end{frame}


\begin{frame}
\frametitle{Voraussetzungen}
\framesubtitle{Was wird benötigt}

\begin{itemize}
\item Aktuelle \TeX\ Installation
\item Python3 Installation, vorzugsweise Anaconda/Winpython
\item Pakete:
\begin{itemize}
	\item numpy
	\item jinja2
	\item pandas
\end{itemize}
\item Diese Folien und Code-Beispiele unter: 
\end{itemize}

\url{https://github.com/UweZiegenhagen/PythonAndLaTeX}

\end{frame}

\begin{frame}[containsverbatim]
\frametitle{Testen der Installation}

\begin{itemize}
	\item Funktionieren die folgenden Befehle?
\end{itemize}

\begin{lstlisting}
import jinja2
import pandas
import numpy
\end{lstlisting}
\end{frame}

\section{Python Grundlagen}

\begin{frame}
\frametitle{Python}
\framesubtitle{~}

\begin{itemize}
	\item Erfunden von Guido van Rossum (Niederlande)
	\item Fokus auf lesbaren und verständlichen Code
	\item \enquote{batteries included} $\Rightarrow$ umfangreiche Standardbibliothek
	\item Mein erster Kontakt mit Python: Downloadskript  für SaveTV
	\item Python2 versus Python3 $\Rightarrow$ Python3
	\item Editor? Ich nutze Spyder3
\end{itemize}
\end{frame}

\begin{frame}[fragile]
\frametitle{Python \enquote{Hello World}}

\begin{lstlisting}[language={Python},caption={Hello World in Python 3.x}]
print('Hello Python')
# Kommentar
a = 123.4
a+=2
print(a+2)

def myFunction(a):
    b = a + a
    return b
    
print(myFunction(2)) # 4
print(myFunction('a')) # 'aa'
\end{lstlisting}

\end{frame}


\begin{frame}[fragile]
\frametitle{Strings, Listen und Tupel}
\framesubtitle{Strings}

\begin{lstlisting}[language={Python},caption={Strings}]
a = 'Hallo'
b = 'Welt'

c = a + ' ' + b
'W' in c # True
print(c[0]) # 'H'
print(c[-1]) # 't'
print(c[1:3]) # 'all'

for i in c:
	print(i)

\end{lstlisting}
\end{frame}

\begin{frame}[fragile]
\frametitle{Strings, Listen und Tupel}
\framesubtitle{Stringfunktionen}

\begin{lstlisting}[language={Python},caption={Strings}]
meinString = 'Hallo Welt'

meinString.upper()
meinString.find('Welt')
meinString.split(' ')
meinString.replace('Welt', 'World')
\end{lstlisting}
\end{frame}


\begin{frame}[fragile]
\frametitle{Strings, Listen und Tupel}
\framesubtitle{Listen}

\begin{lstlisting}[language={Python},caption={Listen}]
beatles = ['John', 'Paul', 'Ringo', 'George']
print(len(beatles))
beatles[0]
beatles.append('')
beatles.index('John')
\end{lstlisting}
\end{frame}


\begin{frame}[fragile]
\frametitle{Strings, Listen und Tupel}
\framesubtitle{Stringfunktionen}

\begin{lstlisting}[language={Python},caption={Strings}]
meinString = 'Hallo Welt'

meinString.upper()
meinString.find('Welt')
meinString.split(' ')
meinString.replace('Welt', 'World')
\end{lstlisting}
\end{frame}


\begin{frame}[fragile]
\frametitle{Strings, Listen und Tupel}
\framesubtitle{Tupel}

\begin{lstlisting}[language={Python},caption={Tupel}]
monate=('Jan', 'Feb', 'Mar', 'Apr', 'Mai')
monate[1]
monate[1:3]
\end{lstlisting}
\end{frame}


\begin{frame}[fragile]
\frametitle{Dictionaries}
\framesubtitle{Key-Value Paare}

\begin{lstlisting}[language={Python},caption={Dictionaries}]
lookup={'EUR':'Euro', 'GBP':'Pound', 'USD':'US-Dollar'}
lookup['EUR']
\end{lstlisting}
\end{frame}

\section{Programmfluss}

\begin{frame}[fragile]
\frametitle{Flusssteuerung}
\framesubtitle{if/then}

\begin{lstlisting}[language={Python},caption={if-then}]
if condition:
	DoThis()
else:
	pass # pass = "Do nothing"
\end{lstlisting}

"condition" kann ein üblicher Boolean Ausdruck sein.

\end{frame}

\begin{frame}[fragile]
\frametitle{Flusssteuerung}
\framesubtitle{for}

\begin{lstlisting}[language={Python},caption={for}]
myString = 'Python'
for c in myString:
	print(c)
\end{lstlisting}

\end{frame}

\begin{frame}[fragile]
\frametitle{Flusssteuerung}
\framesubtitle{while}

\begin{lstlisting}[language={Python},caption={while}]
n = 4
while n>0:
	print(n)
	n-=1
\end{lstlisting}

\end{frame}

\begin{frame}[fragile]
\frametitle{Flusssteuerung}
\framesubtitle{break \& continue}

\begin{lstlisting}[language={Python},caption={break \& continue}]

\end{lstlisting}

\end{frame}


\begin{frame}[fragile]
\frametitle{Funktionen}
\framesubtitle{~}

\begin{lstlisting}[language={Python},caption={Definition von Funktionen}]
def add(a,b):
	return a+b
	
def multiply3(a,b,c=1):
	return a*b*c
	
\end{lstlisting}

\end{frame}


\begin{frame}[fragile]
\frametitle{Ein- und Ausgabe}
\framesubtitle{Kommandozeile}

\begin{lstlisting}[language={Python},caption={Ein- und Ausgabe: Kommandozeile}]
a = input('Erstes Wort')
b = input('Zweites Wort')

print(a, b)
print(a, b, sep='')
print(a, b, sep=':')
\end{lstlisting}
\end{frame}

\begin{frame}[fragile]
\frametitle{Ein- und Ausgabe}
\framesubtitle{Dateien lesen}

\begin{lstlisting}[language={Python},caption={Ein- und Ausgabe: Dateien}]
dateiname = 'datei.txt'

filepointer = open(dateiname,'w') 
# 'r' (read) oder 'a' (append)
for zeile in filepointer:
        print(line)

filepointer.close()
\end{lstlisting}

\begin{itemize}
	\item Um Excel, CSV und ähnliches zu lesen $\Rightarrow$ pandas
\end{itemize}
\end{frame}


\begin{frame}[fragile]
\frametitle{Ein- und Ausgabe}
\framesubtitle{UTF8-Dateien schreiben}

\begin{lstlisting}[language={Python},caption={Ein- und Ausgabe: UTF8 Dateien}]
import io
with io.open(filename,'r',encoding='utf8') as f:
    text = f.read()

with io.open(filename,'w',encoding='utf8') as f:
    f.write(text)
\end{lstlisting}
\end{frame}




\end{document}